\documentclass[12pt]{article}
\usepackage{amsmath}
\usepackage{amssymb}
\usepackage{geometry}
\usepackage{bm}
\usepackage{tensor}
\geometry{margin=1in}

\title{Multipole Electrostatic Formalism in pyEF}
\author{Analysis Module Equations}
\date{}

\begin{document}

\maketitle

\section{Introduction}

This document presents the mathematical formalism implemented in the \texttt{analysis.py} module for computing electrostatic interactions using multipole expansions. The implementation follows the AMOEBA polarizable force field formalism (Ren \& Ponder, J. Phys. Chem. B 2003) and uses tensor-based methods for efficient calculation of multipole-multipole interactions.

\section{Multipole Vector Construction}

\subsection{Multipole Moment Vector}

For each atom, we construct a multipole moment vector $\mathbf{M}$ that contains the charge, dipole, and quadrupole moments:

\begin{equation}
\mathbf{M} =
\begin{cases}
[q] & \text{Order 1 (monopole only)} \\
[q, \mu_x, \mu_y, \mu_z]^T & \text{Order 2 (+ dipole)} \\
[q, \mu_x, \mu_y, \mu_z, Q_{xx}, Q_{xy}, Q_{xz}, Q_{yy}, Q_{yz}]^T & \text{Order 3 (+ quadrupole)}
\end{cases}
\end{equation}

\subsection{Unit Conversions}

The multipole moments are converted from atomic units to SI-based units:

\begin{align}
q &: \text{dimensionless (in units of } e \text{)} \\
\bm{\mu} &= \bm{\mu}_\text{Bohr} \cdot a_0 \quad \text{(units: } e \cdot \text{m)} \\
\mathbf{Q} &= \mathbf{Q}_\text{Bohr} \cdot a_0^2 \quad \text{(units: } e \cdot \text{m}^2 \text{)}
\end{align}

where $a_0 = 0.529177 \times 10^{-10}$ m is the Bohr radius.

\subsection{Traceless Quadrupole Representation}

The quadrupole tensor is symmetric and traceless:

\begin{equation}
\mathbf{Q} = \begin{pmatrix}
Q_{xx} & Q_{xy} & Q_{xz} \\
Q_{xy} & Q_{yy} & Q_{yz} \\
Q_{xz} & Q_{yz} & Q_{zz}
\end{pmatrix}, \quad \text{with } Q_{zz} = -Q_{xx} - Q_{yy}
\end{equation}

Therefore, only 5 independent components are stored: $Q_{xx}, Q_{xy}, Q_{xz}, Q_{yy}, Q_{yz}$.

\section{Interaction Tensor Formalism}

\subsection{General Form}

The interaction tensor $\mathbf{T}_{ij}$ contains all spatial derivatives of the Coulomb potential between atoms $i$ and $j$. The total electrostatic interaction energy is computed as:

\begin{equation}
E_{ij} = \mathbf{M}_i^T \cdot \mathbf{T}_{ij} \cdot \mathbf{M}_j
\end{equation}

\subsection{Distance Vector and Unit Vector}

Given positions $\mathbf{r}_i$ and $\mathbf{r}_j$ in Angstroms:

\begin{align}
\mathbf{r} &= (\mathbf{r}_j - \mathbf{r}_i) \cdot 10^{-10} \text{ m} \\
r &= |\mathbf{r}| \\
\hat{\mathbf{r}} &= \frac{\mathbf{r}}{r}
\end{align}

\subsection{Coulomb Prefactor}

The Coulomb prefactor with dielectric:

\begin{equation}
k = \frac{k_e e^2}{\epsilon_r}
\end{equation}

where:
\begin{itemize}
\item $k_e = 8.987551787 \times 10^9$ N$\cdot$m$^2$/C$^2$ is Coulomb's constant
\item $e = 1.602176634 \times 10^{-19}$ C is the elementary charge
\item $\epsilon_r$ is the relative dielectric constant
\end{itemize}

\subsection{Order 1: Monopole Only}

For monopole-monopole interactions:

\begin{equation}
\mathbf{T}_{ij}^{(1)} = \begin{bmatrix}
\frac{k}{r}
\end{bmatrix}
\end{equation}

Energy:
\begin{equation}
E_{ij}^{(1)} = \frac{k q_i q_j}{r}
\end{equation}

\subsection{Order 2: Monopole + Dipole}

The $4 \times 4$ interaction tensor includes monopole and dipole terms:

\begin{equation}
\mathbf{T}_{ij}^{(2)} = \begin{bmatrix}
\frac{k}{r} & -\frac{k\hat{\mathbf{r}}^T}{r^2} \\[0.5em]
\frac{k\hat{\mathbf{r}}}{r^2} & \frac{k}{r^3}(3\hat{\mathbf{r}}\hat{\mathbf{r}}^T - \mathbf{I})
\end{bmatrix}
\end{equation}

This automatically includes all interactions:
\begin{itemize}
\item Charge-charge: $q_i \times q_j$
\item Charge-dipole: $q_i \times \bm{\mu}_j$ and $\bm{\mu}_i \times q_j$
\item Dipole-dipole: $\bm{\mu}_i \times \bm{\mu}_j$
\end{itemize}

\subsection{Order 3: Monopole + Dipole + Quadrupole}

For order 3, the tensor is $9 \times 9$ and includes all terms up to quadrupole-quadrupole. Key tensor elements:

\subsubsection{Monopole-Monopole}
\begin{equation}
T_{0,0} = \frac{k}{r}
\end{equation}

\subsubsection{Monopole-Dipole}
\begin{align}
T_{0,1:4} &= -\frac{k\hat{\mathbf{r}}}{r^2} \\
T_{1:4,0} &= \frac{k\hat{\mathbf{r}}}{r^2}
\end{align}

\subsubsection{Dipole-Dipole}
\begin{equation}
T_{1:4,1:4} = \frac{k}{r^3}(3\hat{\mathbf{r}}\hat{\mathbf{r}}^T - \mathbf{I})
\end{equation}

\subsubsection{Monopole-Quadrupole}
For each traceless quadrupole component $(\alpha, \beta) \in \{(0,0), (0,1), (0,2), (1,1), (1,2)\}$:

\begin{equation}
T_{0, 4+\text{idx}} = T_{4+\text{idx}, 0} = \frac{k}{r^3}(3\hat{r}_\alpha\hat{r}_\beta - \delta_{\alpha\beta})
\end{equation}

\subsubsection{Dipole-Quadrupole}
For dipole component $i$ and quadrupole component $(\alpha, \beta)$:

\begin{equation}
T_{1+i, 4+\text{idx}} = T_{4+\text{idx}, 1+i} = \frac{k}{r^4}\left[-15\hat{r}_i\hat{r}_\alpha\hat{r}_\beta + 3(\delta_{i\alpha}\hat{r}_\beta + \delta_{i\beta}\hat{r}_\alpha + \delta_{\alpha\beta}\hat{r}_i)\right]
\end{equation}

\subsubsection{Quadrupole-Quadrupole}
For quadrupole components $(\alpha, \beta)$ and $(\gamma, \delta)$:

\begin{multline}
T_{4+\text{idx}_1, 4+\text{idx}_2} = \frac{k}{r^5}\Big[105\hat{r}_\alpha\hat{r}_\beta\hat{r}_\gamma\hat{r}_\delta \\
- 15(\hat{r}_\alpha\hat{r}_\beta\delta_{\gamma\delta} + \hat{r}_\gamma\hat{r}_\delta\delta_{\alpha\beta} + \hat{r}_\alpha\hat{r}_\gamma\delta_{\beta\delta} + \hat{r}_\alpha\hat{r}_\delta\delta_{\beta\gamma} + \hat{r}_\beta\hat{r}_\gamma\delta_{\alpha\delta} + \hat{r}_\beta\hat{r}_\delta\delta_{\alpha\gamma}) \\
+ 3(\delta_{\alpha\beta}\delta_{\gamma\delta} + \delta_{\alpha\gamma}\delta_{\beta\delta} + \delta_{\alpha\delta}\delta_{\beta\gamma})\Big]
\end{multline}

\section{Electrostatic Stabilization Energy}

\subsection{Total Energy Formula}

The total electrostatic stabilization energy between substrate and environment:

\begin{equation}
E_{\text{total}} = \sum_{i \in \text{substrate}} \sum_{j \in \text{environment}} \mathbf{M}_i^T \cdot \mathbf{T}_{ij} \cdot \mathbf{M}_j
\end{equation}

This single equation encompasses all interaction terms:
\begin{itemize}
\item Order 1: charge-charge
\item Order 2: + charge-dipole + dipole-dipole
\item Order 3: + charge-quadrupole + dipole-quadrupole + quadrupole-quadrupole
\end{itemize}

\subsection{Energy Unit Conversion}

The tensor formalism yields energy in Joules. Convert to kcal/mol:

\begin{equation}
E_{\text{kcal/mol}} = E_{\text{J}} \times \frac{N_A}{4184}
\end{equation}

where $N_A = 6.02 \times 10^{23}$ mol$^{-1}$ is Avogadro's number.

\section{Electric Field Calculation}

\subsection{Monopole Electric Field}

The electric field at atom $o$ due to a point charge $q_j$ at distance $\mathbf{r}$:

\begin{equation}
\mathbf{E}_{\text{monopole}} = -\frac{1}{\epsilon_r} k_e e q_j \frac{\mathbf{r}}{r^3}
\end{equation}

where $\mathbf{r} = (\mathbf{r}_j - \mathbf{r}_o) \times 10^{-10}$ m.

\subsection{Multipole Electric Field Components}

The total electric field includes contributions from charge, dipole, and quadrupole moments. For the $x$-component:

\begin{multline}
E_x = \frac{1}{\epsilon_r} k_e \frac{r_x}{r^3} \left[-e q_j + \frac{e}{r^2}(\mu_y r_y + \mu_z r_z)\right] - \frac{1}{3}E_{x,\text{quad}}
\end{multline}

The quadrupole contribution to the $x$-component:

\begin{equation}
E_{x,\text{quad}} = \frac{1}{\epsilon_r} k_e e \frac{r_x}{r^3} \frac{1}{r^4} \sum_{j,k \in \{y,z\}} r_j r_k Q_{jk}
\end{equation}

Similar expressions hold for $E_y$ and $E_z$ components.

\subsection{Complete Atomic Contribution}

For atom $j$ contributing to the field at atom $o$, the total field vector is:

\begin{equation}
\mathbf{E}_j = \frac{1}{\epsilon_r} k_e \left[\frac{-e q_j \mathbf{r}}{r^3} + \frac{e}{r^5}\left(3(\bm{\mu}_j \cdot \mathbf{r})\mathbf{r} - r^2\bm{\mu}_j\right) + \mathbf{E}_{\text{quad}}\right]
\end{equation}

where the quadrupole term involves the contraction:

\begin{equation}
E_{\text{quad},\alpha} = -\frac{1}{3} \frac{e}{r^7} r_\alpha \sum_{\beta,\gamma} r_\beta r_\gamma Q_{\beta\gamma}
\end{equation}

\subsection{Total Electric Field}

The total electric field at atom $o$ is the sum over all atoms $j$ in the charge range:

\begin{equation}
\mathbf{E}_{\text{total}} = \sum_{j \neq o} \mathbf{E}_j
\end{equation}

Electric field units are converted from V/m to V/\AA{} using the conversion factor $10^{-10}$.

\section{QM/MM Support}

The formalism naturally handles QM/MM calculations where substrate and environment can have different multipole orders:

\begin{itemize}
\item QM regions: Use order 2 or 3 (include dipoles/quadrupoles)
\item MM regions: Use order 1 (charges only)
\end{itemize}

The interaction tensor is built with dimension $\max(\text{order}_{\text{substrate}}, \text{order}_{\text{environment}})$, and multipole vectors are zero-padded as needed to maintain proper dimensionality.

\section{Summary of Key Features}

\subsection{Advantages of Tensor Formalism}
\begin{itemize}
\item \textbf{Complete}: Automatically includes all multipole-multipole interaction terms
\item \textbf{Symmetric}: Properly treats both substrate and environment multipoles
\item \textbf{Efficient}: Single matrix multiplication per atom pair
\item \textbf{Extensible}: Easy to add higher-order multipole terms
\item \textbf{Flexible}: Supports mixed multipole orders for QM/MM calculations
\end{itemize}

\subsection{Implementation Notes}
\begin{itemize}
\item All distances in Angstroms are converted to meters for SI unit consistency
\item Charges are treated as dimensionless (in units of $e$)
\item Dielectric constant is applied uniformly to all interactions
\item Traceless quadrupole representation reduces storage and improves numerical stability
\item Automatic fallback to charge-only mode if multipole files are unavailable
\end{itemize}

\end{document}

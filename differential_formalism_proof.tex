\documentclass[12pt]{article}
\usepackage{amsmath}
\usepackage{amssymb}
\usepackage{physics}
\usepackage{geometry}
\usepackage{xcolor}
\geometry{margin=1in}

\title{Differential Formalism: Does $\int \vec{E} \cdot d\vec{\mu}$ Equal $\int \rho V \, dV$?}
\author{}
\date{}

\begin{document}

\maketitle

\section{Question}

For a continuous charge distribution $\rho(\vec{r})$ in an external field $\vec{E}(\vec{r})$ with potential $V(\vec{r})$, does:
\begin{equation}
-\int \vec{E}(\vec{r}) \cdot d\vec{\mu} \stackrel{?}{=} \int \rho(\vec{r}) V(\vec{r}) \, d^3r
\end{equation}

where the differential dipole element is:
\begin{equation}
d\vec{\mu} = \vec{r} \rho(\vec{r}) \, d^3r
\end{equation}

\section{Dimensionality Check}

\textbf{Left side:} $-\int \vec{E}(\vec{r}) \cdot d\vec{\mu}$
\begin{align}
&= -\int \vec{E}(\vec{r}) \cdot \vec{r} \rho(\vec{r}) \, d^3r \\
&\text{Units: } \left[\frac{V}{m}\right] \cdot [m] \cdot \left[\frac{C}{m^3}\right] \cdot [m^3] = [V] \cdot [C] = [J] \quad \checkmark
\end{align}

\textbf{Right side:} $\int \rho(\vec{r}) V(\vec{r}) \, d^3r$
\begin{align}
&\text{Units: } \left[\frac{C}{m^3}\right] \cdot [V] \cdot [m^3] = [C] \cdot [V] = [J] \quad \checkmark
\end{align}

\textbf{Dimensionality is correct!}

\section{Uniform Field: Do They Match?}

\subsection{Setup}
\begin{itemize}
    \item Uniform field: $\vec{E}(\vec{r}) = \vec{F}$ (constant)
    \item Potential: $V(\vec{r}) = V_0 - \vec{F} \cdot \vec{r}$
    \item Charge distribution: $\rho(\vec{r})$
    \item Total charge: $Q = \int \rho(\vec{r}) \, d^3r$
\end{itemize}

\subsection{Right Side: Standard Formulation}
\begin{align}
E_{\text{standard}} &= \int \rho(\vec{r}) V(\vec{r}) \, d^3r \\
&= \int \rho(\vec{r}) (V_0 - \vec{F} \cdot \vec{r}) \, d^3r \\
&= V_0 \int \rho(\vec{r}) \, d^3r - \vec{F} \cdot \int \vec{r} \rho(\vec{r}) \, d^3r \\
&= V_0 Q - \vec{F} \cdot \vec{\mu}_{\text{total}}
\end{align}

For neutral system ($Q = 0$):
\begin{equation}
E_{\text{standard}} = -\vec{F} \cdot \vec{\mu}_{\text{total}}
\end{equation}

\subsection{Left Side: Differential Dipole Formulation}
\begin{align}
E_{\text{dipole}} &= -\int \vec{E}(\vec{r}) \cdot \vec{r} \rho(\vec{r}) \, d^3r \\
&= -\int \vec{F} \cdot \vec{r} \rho(\vec{r}) \, d^3r \\
&= -\vec{F} \cdot \int \vec{r} \rho(\vec{r}) \, d^3r \\
&= -\vec{F} \cdot \vec{\mu}_{\text{total}}
\end{align}

\subsection{Result}
\begin{equation}
\boxed{E_{\text{standard}} = E_{\text{dipole}} = -\vec{F} \cdot \vec{\mu}_{\text{total}}}
\end{equation}

\textbf{For uniform fields, they are IDENTICAL!}

\section{Non-Uniform Field: Do They Match?}

\subsection{Setup}
Consider a point charge creating a non-uniform field:
\begin{itemize}
    \item External point charge $Q_{\text{ext}}$ at position $\vec{R}$
    \item Field: $\vec{E}(\vec{r}) = k_e Q_{\text{ext}} \frac{\vec{r} - \vec{R}}{|\vec{r} - \vec{R}|^3}$
    \item Potential: $V(\vec{r}) = k_e \frac{Q_{\text{ext}}}{|\vec{r} - \vec{R}|}$
\end{itemize}

\subsection{Standard Formulation}
\begin{equation}
E_{\text{standard}} = \int \rho(\vec{r}) V(\vec{r}) \, d^3r = k_e Q_{\text{ext}} \int \frac{\rho(\vec{r})}{|\vec{r} - \vec{R}|} \, d^3r
\end{equation}

This is the \textbf{exact} energy.

\subsection{Differential Dipole Formulation}
\begin{align}
E_{\text{dipole}} &= -\int \vec{E}(\vec{r}) \cdot \vec{r} \rho(\vec{r}) \, d^3r \\
&= -k_e Q_{\text{ext}} \int \frac{\vec{r} - \vec{R}}{|\vec{r} - \vec{R}|^3} \cdot \vec{r} \, \rho(\vec{r}) \, d^3r \\
&= -k_e Q_{\text{ext}} \int \frac{\vec{r} \cdot \vec{r} - \vec{R} \cdot \vec{r}}{|\vec{r} - \vec{R}|^3} \rho(\vec{r}) \, d^3r \\
&= -k_e Q_{\text{ext}} \int \frac{|\vec{r}|^2 - \vec{R} \cdot \vec{r}}{|\vec{r} - \vec{R}|^3} \rho(\vec{r}) \, d^3r
\end{align}

\subsection{Comparison}

These are \textbf{NOT equal} in general!

\begin{equation}
\int \frac{\rho(\vec{r})}{|\vec{r} - \vec{R}|} \, d^3r \neq \int \frac{|\vec{r}|^2 - \vec{R} \cdot \vec{r}}{|\vec{r} - \vec{R}|^3} \rho(\vec{r}) \, d^3r
\end{equation}

\textbf{The formulations differ for non-uniform fields!}

\section{Physical Interpretation}

\subsection{What is $-\int \vec{E}(\vec{r}) \cdot \vec{r} \rho(\vec{r}) \, d^3r$?}

This is NOT the correct energy! It's related to the energy but includes extra terms.

The issue is that $\vec{E}(\vec{r}) \cdot \vec{r}$ is not equal to $V(\vec{r})$ in general.

\subsection{Correct Relationship}

The fundamental relationship is:
\begin{equation}
\vec{E}(\vec{r}) = -\nabla V(\vec{r})
\end{equation}

In integral form:
\begin{equation}
V(\vec{r}) = V(\vec{r}_0) - \int_{\vec{r}_0}^{\vec{r}} \vec{E}(\vec{r}') \cdot d\vec{r}'
\end{equation}

This is a \textbf{line integral} along a path from $\vec{r}_0$ to $\vec{r}$, NOT a simple dot product $\vec{E}(\vec{r}) \cdot \vec{r}$.

\subsection{When Do They Match?}

For a uniform field $\vec{E} = \vec{F}$:
\begin{align}
V(\vec{r}) &= V_0 - \int_0^{\vec{r}} \vec{F} \cdot d\vec{r}' \\
&= V_0 - \vec{F} \cdot \vec{r}
\end{align}

Therefore:
\begin{equation}
-\vec{F} \cdot \vec{r} = V(\vec{r}) - V_0
\end{equation}

So for uniform fields:
\begin{align}
-\int \vec{F} \cdot \vec{r} \rho(\vec{r}) \, d^3r &= \int \rho(\vec{r}) [V(\vec{r}) - V_0] \, d^3r \\
&= \int \rho(\vec{r}) V(\vec{r}) \, d^3r - V_0 \int \rho(\vec{r}) \, d^3r
\end{align}

For neutral systems ($\int \rho \, d^3r = 0$):
\begin{equation}
-\int \vec{F} \cdot \vec{r} \rho(\vec{r}) \, d^3r = \int \rho(\vec{r}) V(\vec{r}) \, d^3r
\end{equation}

\textbf{They match only for uniform fields!}

\section{Bond Dipole in Differential Form}

\subsection{1D Charge Distribution Along Bond}

For a bond from $\vec{r}_A$ to $\vec{r}_B$ with linear charge density $\lambda(s)$:

Position along bond: $\vec{r}(s) = \vec{r}_A + s\hat{n}$, where $s \in [0, L]$ and $\hat{n} = \frac{\vec{r}_B - \vec{r}_A}{L}$

\subsubsection{Standard Formulation}
\begin{equation}
E_{\text{bond,standard}} = \int_0^L \lambda(s) V(\vec{r}(s)) \, ds
\end{equation}

\subsubsection{Differential Dipole Formulation}
\begin{equation}
E_{\text{bond,dipole}} = -\int_0^L \vec{E}(\vec{r}(s)) \cdot \vec{r}(s) \lambda(s) \, ds
\end{equation}

\subsection{Uniform Field}

For $\vec{E} = \vec{F}$ and $V(\vec{r}) = V_0 - \vec{F} \cdot \vec{r}$:

\textbf{Standard:}
\begin{align}
E_{\text{bond,standard}} &= \int_0^L \lambda(s) (V_0 - \vec{F} \cdot \vec{r}(s)) \, ds \\
&= V_0 \int_0^L \lambda(s) \, ds - \vec{F} \cdot \int_0^L \vec{r}(s) \lambda(s) \, ds \\
&= V_0 Q_{\text{bond}} - \vec{F} \cdot \vec{\mu}_{\text{bond}}
\end{align}

\textbf{Dipole:}
\begin{align}
E_{\text{bond,dipole}} &= -\int_0^L \vec{F} \cdot \vec{r}(s) \lambda(s) \, ds \\
&= -\vec{F} \cdot \int_0^L \vec{r}(s) \lambda(s) \, ds \\
&= -\vec{F} \cdot \vec{\mu}_{\text{bond}}
\end{align}

For neutral bond ($Q_{\text{bond}} = 0$):
\begin{equation}
\boxed{E_{\text{bond,standard}} = E_{\text{bond,dipole}} = -\vec{F} \cdot \vec{\mu}_{\text{bond}}}
\end{equation}

\subsection{Non-Uniform Field}

For non-uniform fields:
\begin{equation}
\int_0^L \lambda(s) V(\vec{r}(s)) \, ds \neq -\int_0^L \vec{E}(\vec{r}(s)) \cdot \vec{r}(s) \lambda(s) \, ds
\end{equation}

The dipole formulation is an \textbf{approximation} that misses higher-order multipole contributions.

\section{Correct Differential Formulation}

\subsection{The Key Issue}

The problem with $-\int \vec{E} \cdot \vec{r} \rho \, d^3r$ is that it uses $\vec{r}$ (position relative to origin), not the proper potential.

\subsection{Alternative: Integration by Parts}

For a neutral distribution, we can relate the two formulations:

Starting from $E = \int \rho V \, d^3r$ with $\vec{E} = -\nabla V$:

Using integration by parts (and assuming $\rho$ vanishes at infinity):
\begin{align}
E &= \int \rho(\vec{r}) V(\vec{r}) \, d^3r \\
&= -\int \rho(\vec{r}) \int_{\vec{r}_0}^{\vec{r}} \vec{E}(\vec{r}') \cdot d\vec{r}' \, d^3r + V(\vec{r}_0) \int \rho \, d^3r
\end{align}

For neutral systems ($\int \rho \, d^3r = 0$):
\begin{equation}
E = -\int \rho(\vec{r}) \left[\int_{\vec{r}_0}^{\vec{r}} \vec{E}(\vec{r}') \cdot d\vec{r}'\right] d^3r
\end{equation}

This is NOT the same as $-\int \vec{E}(\vec{r}) \cdot \vec{r} \rho(\vec{r}) \, d^3r$ unless the field is uniform.

\section{Summary and Conclusions}

\begin{enumerate}
    \item \textbf{Dimensionality}: The formulation $-\int \vec{E}(\vec{r}) \cdot d\vec{\mu}$ with $d\vec{\mu} = \vec{r} \rho(\vec{r}) d^3r$ is dimensionally correct ✓

    \item \textbf{Uniform fields}: For uniform $\vec{E} = \vec{F}$ and neutral systems:
    \begin{equation}
    \boxed{\int \rho V \, d^3r = -\int \vec{F} \cdot \vec{r} \rho \, d^3r = -\vec{F} \cdot \vec{\mu}_{\text{total}}}
    \end{equation}
    The formulations are \textbf{IDENTICAL} ✓

    \item \textbf{Non-uniform fields}: The formulations are \textbf{DIFFERENT}:
    \begin{equation}
    \int \rho V \, d^3r \neq -\int \vec{E}(\vec{r}) \cdot \vec{r} \rho \, d^3r
    \end{equation}

    \item \textbf{Physical reason}: $V(\vec{r}) \neq -\vec{E}(\vec{r}) \cdot \vec{r}$ for non-uniform fields

    Only for uniform fields: $V(\vec{r}) = V_0 - \vec{F} \cdot \vec{r}$, so they match.

    \item \textbf{Bond formulation}: Same conclusions apply to 1D distributions along bonds:
    \begin{itemize}
        \item Uniform field: $\int \lambda V \, ds = -\int \vec{F} \cdot \vec{r} \lambda \, ds$ ✓
        \item Non-uniform field: $\int \lambda V \, ds \neq -\int \vec{E} \cdot \vec{r} \lambda \, ds$ ✗
    \end{itemize}

    \item \textbf{Recommendation}: Always use $\boxed{E = \int \rho(\vec{r}) V(\vec{r}) \, d^3r}$ as ground truth
\end{enumerate}

\section{Answer to Original Question}

\textbf{Question}: If we take the integral of the electric field over the bond dipole as the total stabilization, would we get the same thing as the potential and electron density?

\textbf{Answer}:
\begin{itemize}
    \item \textcolor{green}{\textbf{YES}} for uniform electric fields on neutral systems
    \item \textcolor{red}{\textbf{NO}} for non-uniform electric fields
    \item The exact formula $\int \rho V \, dV$ is always correct
    \item The dipole formula $-\int \vec{E} \cdot \vec{r} \rho \, dV$ is an approximation for non-uniform fields
\end{itemize}

The bond dipole \textbf{should} be in differential formalism as:
\begin{equation}
d\vec{\mu} = \vec{r}(s) \lambda(s) \, ds \quad \text{(for 1D)} \quad \text{or} \quad d\vec{\mu} = \vec{r} \rho(\vec{r}) \, d^3r \quad \text{(for 3D)}
\end{equation}

\end{document}

\documentclass[12pt]{article}
\usepackage{amsmath}
\usepackage{amssymb}
\usepackage{physics}
\usepackage{geometry}
\usepackage{xcolor}
\geometry{margin=1in}

\title{Line Integral Formalism: E-field$\cdot$Dipole vs. Potential$\cdot$Charges}
\author{}
\date{}

\begin{document}

\maketitle

\section{Question}

For charges distributed along bonds with dipole moments $\vec{\mu}_{\text{bond}}$, does:
\begin{equation}
-\int_{\text{bond}} \vec{E}(\vec{r}) \cdot d\vec{l} \times \text{(charge distribution)} \stackrel{?}{=} \int_{\text{bond}} \lambda(s) V(\vec{r}(s)) \, ds
\end{equation}

And for discrete bond dipoles, is the energy just:
\begin{equation}
E = -\sum_i \vec{\mu}_i \cdot \vec{E}_i \quad ?
\end{equation}

\section{Review: Fundamental Relationships}

\subsection{Electric Field and Potential}

The fundamental relationship is:
\begin{equation}
\vec{E}(\vec{r}) = -\nabla V(\vec{r})
\end{equation}

In line integral form:
\begin{equation}
V(\vec{r}_B) - V(\vec{r}_A) = -\int_A^B \vec{E}(\vec{r}) \cdot d\vec{l}
\end{equation}

Or equivalently:
\begin{equation}
\boxed{V(\vec{r}) = V_0 - \int_{\vec{r}_0}^{\vec{r}} \vec{E}(\vec{r}') \cdot d\vec{l}'}
\end{equation}

\subsection{Energy of Charges in External Potential}

The exact energy is:
\begin{equation}
E_{\text{exact}} = \int \rho(\vec{r}) V(\vec{r}) \, d^3r
\end{equation}

For 1D distribution along a bond:
\begin{equation}
E_{\text{bond,exact}} = \int_0^L \lambda(s) V(\vec{r}(s)) \, ds
\end{equation}

\section{Line Integral Formulation for Bond Energy}

\subsection{Setup: Bond from A to B}

Consider a bond from $\vec{r}_A$ to $\vec{r}_B$ with:
\begin{itemize}
    \item Position along bond: $\vec{r}(s) = \vec{r}_A + s\hat{n}$, where $s \in [0, L]$
    \item Unit vector: $\hat{n} = \frac{\vec{r}_B - \vec{r}_A}{L}$
    \item Linear charge density: $\lambda(s)$
    \item Total charge: $Q_{\text{bond}} = \int_0^L \lambda(s) \, ds$
    \item Dipole moment: $\vec{\mu}_{\text{bond}} = \int_0^L \vec{r}(s) \lambda(s) \, ds$
\end{itemize}

\subsection{Exact Energy Using Potential}

\begin{equation}
E_{\text{exact}} = \int_0^L \lambda(s) V(\vec{r}(s)) \, ds
\end{equation}

\subsection{Line Integral Formulation}

Using $V(\vec{r}(s)) = V_0 - \int_{\vec{r}_0}^{\vec{r}(s)} \vec{E}(\vec{r}') \cdot d\vec{l}'$:

\begin{align}
E &= \int_0^L \lambda(s) \left[V_0 - \int_{\vec{r}_0}^{\vec{r}(s)} \vec{E}(\vec{r}') \cdot d\vec{l}'\right] ds \\
&= V_0 \int_0^L \lambda(s) \, ds - \int_0^L \lambda(s) \left[\int_{\vec{r}_0}^{\vec{r}(s)} \vec{E}(\vec{r}') \cdot d\vec{l}'\right] ds \\
&= V_0 Q_{\text{bond}} - \int_0^L \lambda(s) \left[\int_{\vec{r}_0}^{\vec{r}(s)} \vec{E}(\vec{r}') \cdot d\vec{l}'\right] ds
\end{align}

For neutral bonds ($Q_{\text{bond}} = 0$):
\begin{equation}
\boxed{E = -\int_0^L \lambda(s) \left[\int_{\vec{r}_0}^{\vec{r}(s)} \vec{E}(\vec{r}') \cdot d\vec{l}'\right] ds}
\end{equation}

This is a \textbf{nested integral}: we integrate the field along a path to each point $\vec{r}(s)$, weighted by the charge density at that point.

\section{Uniform Field: Simple Case}

\subsection{Setup}
\begin{itemize}
    \item Uniform field: $\vec{E}(\vec{r}) = \vec{F}$ (constant)
    \item Potential: $V(\vec{r}) = V_0 - \vec{F} \cdot \vec{r}$
\end{itemize}

\subsection{Exact Energy}

\begin{align}
E_{\text{exact}} &= \int_0^L \lambda(s) V(\vec{r}(s)) \, ds \\
&= \int_0^L \lambda(s) (V_0 - \vec{F} \cdot \vec{r}(s)) \, ds \\
&= V_0 Q_{\text{bond}} - \vec{F} \cdot \int_0^L \vec{r}(s) \lambda(s) \, ds \\
&= V_0 Q_{\text{bond}} - \vec{F} \cdot \vec{\mu}_{\text{bond}}
\end{align}

For neutral bond:
\begin{equation}
E_{\text{exact}} = -\vec{F} \cdot \vec{\mu}_{\text{bond}}
\end{equation}

\subsection{Line Integral Formulation}

For uniform $\vec{E} = \vec{F}$:
\begin{align}
\int_{\vec{r}_0}^{\vec{r}(s)} \vec{F} \cdot d\vec{l}' &= \vec{F} \cdot \int_{\vec{r}_0}^{\vec{r}(s)} d\vec{l}' \\
&= \vec{F} \cdot [\vec{r}(s) - \vec{r}_0]
\end{align}

Therefore:
\begin{align}
E &= V_0 Q_{\text{bond}} - \int_0^L \lambda(s) \vec{F} \cdot [\vec{r}(s) - \vec{r}_0] \, ds \\
&= V_0 Q_{\text{bond}} - \vec{F} \cdot \int_0^L \vec{r}(s) \lambda(s) \, ds + \vec{F} \cdot \vec{r}_0 \int_0^L \lambda(s) \, ds \\
&= V_0 Q_{\text{bond}} - \vec{F} \cdot \vec{\mu}_{\text{bond}} + \vec{F} \cdot \vec{r}_0 Q_{\text{bond}}
\end{align}

For neutral bond ($Q_{\text{bond}} = 0$):
\begin{equation}
\boxed{E = -\vec{F} \cdot \vec{\mu}_{\text{bond}}}
\end{equation}

\textbf{Perfect match!} For uniform fields, the line integral formulation gives the exact answer.

\section{Discrete Dipole Approximation}

\subsection{Simple Dipole Formula}

The standard approximation is:
\begin{equation}
E_{\text{dipole}} = -\vec{\mu}_{\text{bond}} \cdot \vec{E}_{\text{center}}
\end{equation}

where $\vec{E}_{\text{center}}$ is evaluated at some representative point (e.g., bond center).

\subsection{Is This Correct?}

\subsubsection{For Uniform Fields}

If $\vec{E}(\vec{r}) = \vec{F}$ everywhere:
\begin{equation}
E_{\text{dipole}} = -\vec{\mu}_{\text{bond}} \cdot \vec{F} = -\vec{F} \cdot \vec{\mu}_{\text{bond}}
\end{equation}

This \textbf{exactly} matches the exact result! ✓

\subsubsection{For Non-Uniform Fields}

The exact energy is:
\begin{equation}
E_{\text{exact}} = \int_0^L \lambda(s) V(\vec{r}(s)) \, ds
\end{equation}

The dipole approximation:
\begin{equation}
E_{\text{dipole}} = -\vec{\mu}_{\text{bond}} \cdot \vec{E}(\vec{r}_{\text{center}})
\end{equation}

These are \textbf{NOT equal} in general. The dipole approximation misses:
\begin{itemize}
    \item Higher-order multipole contributions (quadrupole, octupole, etc.)
    \item Spatial variation of the field across the charge distribution
\end{itemize}

\section{Point Charge Example: Non-Uniform Field}

\subsection{Setup}

External point charge $Q_{\text{ext}}$ at origin:
\begin{itemize}
    \item Field: $\vec{E}(\vec{r}) = k_e Q_{\text{ext}} \frac{\vec{r}}{r^3}$
    \item Potential: $V(\vec{r}) = k_e \frac{Q_{\text{ext}}}{r}$
\end{itemize}

Bond from $\vec{r}_A$ to $\vec{r}_B$ with simple charge distribution: $+q$ at B, $-q$ at A.

Dipole moment: $\vec{\mu}_{\text{bond}} = q(\vec{r}_B - \vec{r}_A)$

\subsection{Exact Energy}

\begin{align}
E_{\text{exact}} &= (-q) V(\vec{r}_A) + (+q) V(\vec{r}_B) \\
&= -q \cdot k_e \frac{Q_{\text{ext}}}{r_A} + q \cdot k_e \frac{Q_{\text{ext}}}{r_B} \\
&= k_e q Q_{\text{ext}} \left(\frac{1}{r_B} - \frac{1}{r_A}\right)
\end{align}

\subsection{Dipole Approximation}

At bond center $\vec{r}_c = \frac{\vec{r}_A + \vec{r}_B}{2}$:
\begin{equation}
\vec{E}(\vec{r}_c) = k_e Q_{\text{ext}} \frac{\vec{r}_c}{r_c^3}
\end{equation}

\begin{equation}
E_{\text{dipole}} = -\vec{\mu}_{\text{bond}} \cdot \vec{E}(\vec{r}_c) = -q(\vec{r}_B - \vec{r}_A) \cdot k_e Q_{\text{ext}} \frac{\vec{r}_c}{r_c^3}
\end{equation}

\subsection{Comparison}

In general:
\begin{equation}
k_e q Q_{\text{ext}} \left(\frac{1}{r_B} - \frac{1}{r_A}\right) \neq -q(\vec{r}_B - \vec{r}_A) \cdot k_e Q_{\text{ext}} \frac{\vec{r}_c}{r_c^3}
\end{equation}

The dipole approximation is only accurate when the bond is:
\begin{itemize}
    \item Small compared to distance from field source
    \item In a slowly-varying field region
\end{itemize}

\section{Discrete Sum of Bond Dipoles}

\subsection{The Question}

For a molecule with multiple bonds, each with dipole $\vec{\mu}_i$, is the total energy:
\begin{equation}
E_{\text{total}} = -\sum_i \vec{\mu}_i \cdot \vec{E}_i \quad ?
\end{equation}

\subsection{Answer}

\textbf{It depends on the field!}

\subsubsection{Uniform Field: YES}

If $\vec{E}(\vec{r}) = \vec{F}$ everywhere:
\begin{align}
E_{\text{exact}} &= -\vec{F} \cdot \sum_i \vec{\mu}_i \\
&= -\sum_i \vec{F} \cdot \vec{\mu}_i \\
&= -\sum_i \vec{\mu}_i \cdot \vec{E}_i
\end{align}

Since $\vec{E}_i = \vec{F}$ for all $i$, this is \textbf{exact}! ✓

\subsubsection{Non-Uniform Field: APPROXIMATION}

For non-uniform fields, this is an approximation that:
\begin{itemize}
    \item Assumes each bond's energy $\approx -\vec{\mu}_i \cdot \vec{E}_i$ (dipole approximation)
    \item Neglects higher-order multipole terms
    \item Becomes more accurate when:
    \begin{itemize}
        \item Bonds are small compared to field variation length scale
        \item Field gradient is weak across each bond
    \end{itemize}
\end{itemize}

\subsection{Discrete Sum Formula}

For discrete bond dipoles in a field $\vec{E}(\vec{r})$:

\begin{equation}
\boxed{E_{\text{dipole}} = -\sum_{i=1}^{N_{\text{bonds}}} \vec{\mu}_i \cdot \vec{E}(\vec{r}_i)}
\end{equation}

where:
\begin{itemize}
    \item $\vec{\mu}_i$ = dipole moment of bond $i$
    \item $\vec{r}_i$ = position where field is evaluated (typically bond center or midpoint)
    \item $\vec{E}(\vec{r}_i)$ = electric field at that position
\end{itemize}

\textbf{Yes, it's just a sum of dot products!} But remember:
\begin{itemize}
    \item This is \textbf{exact} for uniform fields
    \item This is an \textbf{approximation} for non-uniform fields
\end{itemize}

\section{Line Integral Along Bond Path}

\subsection{Alternative Formulation}

Instead of evaluating $\vec{E}$ at a single point, we could integrate along the bond:

\begin{equation}
E_{\text{bond}} = -\int_{\vec{r}_A}^{\vec{r}_B} \vec{E}(\vec{r}) \cdot d\vec{l}
\end{equation}

But wait—this is just $V(\vec{r}_B) - V(\vec{r}_A)$!

For a simple dipole with charges $\pm q$ at the ends:
\begin{align}
E_{\text{exact}} &= (-q) V(\vec{r}_A) + (+q) V(\vec{r}_B) \\
&= q[V(\vec{r}_B) - V(\vec{r}_A)] \\
&= -q \int_{\vec{r}_A}^{\vec{r}_B} \vec{E}(\vec{r}) \cdot d\vec{l}
\end{align}

\subsection{Comparison with Dipole Approximation}

\textbf{Line integral (exact):}
\begin{equation}
E = -q \int_{\vec{r}_A}^{\vec{r}_B} \vec{E}(\vec{r}) \cdot d\vec{l}
\end{equation}

\textbf{Dipole approximation:}
\begin{equation}
E_{\text{dipole}} = -q \vec{E}(\vec{r}_c) \cdot (\vec{r}_B - \vec{r}_A)
\end{equation}

The dipole approximation assumes $\vec{E}$ is constant (equals $\vec{E}(\vec{r}_c)$) along the entire path!

\subsection{When Are They Equal?}

Using mean value theorem for line integrals:
\begin{equation}
\int_{\vec{r}_A}^{\vec{r}_B} \vec{E}(\vec{r}) \cdot d\vec{l} = \vec{E}(\vec{r}_*) \cdot (\vec{r}_B - \vec{r}_A)
\end{equation}

for some point $\vec{r}_*$ on the bond.

If we choose $\vec{r}_c = \vec{r}_*$, they match! But in general, we don't know where $\vec{r}_*$ is without knowing $\vec{E}(\vec{r})$.

For uniform fields: $\vec{r}_*$ can be anywhere (field is constant), so choosing the center works perfectly.

For slowly-varying fields: $\vec{r}_* \approx \vec{r}_c$ (center), so the approximation is good.

\section{Summary and Conclusions}

\begin{enumerate}
    \item \textbf{Line integral relationship}: The exact energy can be written as:
    \begin{equation}
    E = -\int_0^L \lambda(s) \left[\int_{\vec{r}_0}^{\vec{r}(s)} \vec{E}(\vec{r}') \cdot d\vec{l}'\right] ds
    \end{equation}
    This is a nested integral—NOT a simple line integral along the bond.

    \item \textbf{For uniform fields}: Everything simplifies beautifully:
    \begin{equation}
    \boxed{E = -\vec{F} \cdot \vec{\mu}_{\text{total}} = -\sum_i \vec{\mu}_i \cdot \vec{F}}
    \end{equation}
    The discrete dipole sum is \textbf{exact}! ✓

    \item \textbf{Discrete dipole formula}: Yes, it's just a sum of dot products:
    \begin{equation}
    \boxed{E = -\sum_i \vec{\mu}_i \cdot \vec{E}(\vec{r}_i)}
    \end{equation}
    where $\vec{r}_i$ is typically the bond center.

    \item \textbf{For non-uniform fields}: This is an \textbf{approximation} that:
    \begin{itemize}
        \item Works well when bonds are small compared to field variation
        \item Misses higher-order multipole contributions
        \item Can be improved by evaluating $\vec{E}$ more carefully
    \end{itemize}

    \item \textbf{Line integral for two-charge dipole}:
    \begin{equation}
    E_{\text{exact}} = -q \int_{\vec{r}_A}^{\vec{r}_B} \vec{E}(\vec{r}) \cdot d\vec{l}
    \end{equation}
    The dipole approximation replaces this integral with $\vec{E}(\vec{r}_c) \cdot L\hat{n}$.

    \item \textbf{Recommendation}:
    \begin{itemize}
        \item For \textbf{uniform fields}: Use $E = -\sum_i \vec{\mu}_i \cdot \vec{F}$ (exact)
        \item For \textbf{non-uniform fields}: Use $E = \int \rho V \, dV$ or $E = \sum_i q_i V(\vec{r}_i)$ (exact)
        \item The dipole sum $-\sum_i \vec{\mu}_i \cdot \vec{E}_i$ is a useful approximation but not exact
    \end{itemize}
\end{enumerate}

\section{Answer to Your Questions}

\subsection{Does line integral of E-field$\cdot$dipole equal potential$\cdot$charges?}

\textbf{Answer}:
\begin{itemize}
    \item \textcolor{green}{\textbf{YES}} for uniform electric fields (exact)
    \item \textcolor{orange}{\textbf{APPROXIMATELY}} for slowly-varying non-uniform fields
    \item \textcolor{red}{\textbf{NO}} in general for strongly non-uniform fields
\end{itemize}

The exact formulation using $\int \rho V \, dV$ is always correct and should be used as ground truth.

\subsection{Is discrete version just sum of dot products?}

\textbf{Answer}: \textcolor{green}{\textbf{YES!}}

\begin{equation}
\boxed{E_{\text{discrete}} = -\sum_{i=1}^{N_{\text{bonds}}} \vec{\mu}_i \cdot \vec{E}(\vec{r}_i)}
\end{equation}

This is:
\begin{itemize}
    \item \textbf{Exact} for uniform fields
    \item A good \textbf{approximation} for non-uniform fields when bonds are small
    \item Simple to compute: just dot products!
\end{itemize}

Each term $-\vec{\mu}_i \cdot \vec{E}(\vec{r}_i)$ represents the interaction energy of dipole $i$ with the external field evaluated at position $\vec{r}_i$ (typically the bond center or midpoint).

\end{document}

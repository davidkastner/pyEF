\documentclass[12pt]{article}
\usepackage{amsmath}
\usepackage{amssymb}
\usepackage{physics}
\usepackage{geometry}
\geometry{margin=1in}

\title{Proof: Dipole-Field Formulation for Electrostatic Stabilization Energy}
\author{}
\date{}

\begin{document}

\maketitle

\section{Introduction}

We examine when the bond dipole-electric field formulation equals the exact charge-potential formula for electrostatic stabilization energy:
\begin{equation}
E_{\text{exact}} = \sum_i q_i V(\vec{r}_i)
\end{equation}

We consider different conventions for defining bond dipoles and electric field projections.

\section{Uniform Electric Field: Exact Result}

\subsection{Setup}
\begin{itemize}
    \item Uniform electric field: $\vec{F}$ = constant
    \item Electrostatic potential: $V(\vec{r}) = V_0 - \vec{F} \cdot \vec{r}$
    \item Molecule with charges $\{q_i\}$ at positions $\{\vec{r}_i\}$
    \item Charge conservation: $\sum_i q_i = Q_{\text{total}}$
\end{itemize}

\subsection{Exact Energy}
\begin{align}
E_{\text{exact}} &= \sum_i q_i V(\vec{r}_i) \\
&= \sum_i q_i (V_0 - \vec{F} \cdot \vec{r}_i) \\
&= V_0 \sum_i q_i - \vec{F} \cdot \sum_i q_i \vec{r}_i \\
&= V_0 Q_{\text{total}} - \vec{F} \cdot \vec{\mu}_{\text{total}}
\end{align}

For a \textbf{neutral molecule} ($Q_{\text{total}} = 0$):
\begin{equation}
\boxed{E_{\text{exact}} = -\vec{F} \cdot \vec{\mu}_{\text{total}}}
\end{equation}

where $\vec{\mu}_{\text{total}} = \sum_i q_i \vec{r}_i$ is the total molecular dipole moment.

\textbf{Key Result:} For uniform fields, the dipole formula using the \textit{total molecular dipole} is EXACT.

\subsection{Bond Dipole Formulation}

For a bond between atoms $i$ and $j$, we consider different dipole definitions:

\subsubsection{Convention 1: Midpoint Reference}
\begin{equation}
\vec{\mu}_{ij}^{(1)} = (q_j - q_i) \frac{\vec{r}_j - \vec{r}_i}{2}
\end{equation}

\subsubsection{Convention 2: Full Bond Length}
\begin{equation}
\vec{\mu}_{ij}^{(2)} = (q_j - q_i) (\vec{r}_j - \vec{r}_i)
\end{equation}

\subsubsection{Convention 3: Single Charge Reference}
\begin{equation}
\vec{\mu}_{ij}^{(3)} = q_j (\vec{r}_j - \vec{r}_i)
\end{equation}

\subsection{Proof: Convention 1 with Averaged E-field}

For uniform field, $\vec{E}(\vec{r}) = \vec{F}$ everywhere.

The potential difference across bond $(i,j)$:
\begin{equation}
V_j - V_i = (V_0 - \vec{F} \cdot \vec{r}_j) - (V_0 - \vec{F} \cdot \vec{r}_i) = -\vec{F} \cdot (\vec{r}_j - \vec{r}_i)
\end{equation}

Electric field averaged at bond endpoints:
\begin{equation}
\vec{E}_{\text{avg}} = \frac{\vec{E}(\vec{r}_i) + \vec{E}(\vec{r}_j)}{2} = \frac{\vec{F} + \vec{F}}{2} = \vec{F}
\end{equation}

Using the approximation $V_j - V_i \approx -\vec{E}_{\text{avg}} \cdot (\vec{r}_j - \vec{r}_i) \cdot |\vec{r}_j - \vec{r}_i|$:

For uniform field, this simplifies to:
\begin{equation}
V_j - V_i = -\vec{F} \cdot (\vec{r}_j - \vec{r}_i)
\end{equation}

Bond dipole energy:
\begin{align}
E_{ij} &= -\vec{\mu}_{ij}^{(1)} \cdot \vec{F} \\
&= -(q_j - q_i) \frac{\vec{r}_j - \vec{r}_i}{2} \cdot \vec{F} \\
&= \frac{1}{2}(q_j - q_i)(V_j - V_i)
\end{align}

Summing over all bonds (linear chain 0-1-2):
\begin{align}
E_{\text{dipole}} &= \frac{1}{2}[(q_1 - q_0)(V_1 - V_0) + (q_2 - q_1)(V_2 - V_1)] \\
&= \frac{1}{2}[q_1 V_1 - q_1 V_0 - q_0 V_1 + q_0 V_0 + q_2 V_2 - q_2 V_1 - q_1 V_2 + q_1 V_1]
\end{align}

Grouping by potential:
\begin{equation}
E_{\text{dipole}} = \frac{1}{2}[(q_0 - q_1)V_0 + (2q_1 - q_0 - q_2)V_1 + (q_2 - q_1)V_2]
\end{equation}

Compare to exact:
\begin{equation}
E_{\text{exact}} = q_0 V_0 + q_1 V_1 + q_2 V_2
\end{equation}

\textbf{Mismatch:}
\begin{align}
E_{\text{exact}} - E_{\text{dipole}} &= q_0 V_0 + q_1 V_1 + q_2 V_2 - \frac{1}{2}[(q_0 - q_1)V_0 + (2q_1 - q_0 - q_2)V_1 + (q_2 - q_1)V_2] \\
&= \frac{1}{2}[(2q_0 - q_0 + q_1)V_0 + (2q_1 - 2q_1 + q_0 + q_2)V_1 + (2q_2 - q_2 + q_1)V_2] \\
&= \frac{1}{2}[(q_0 + q_1)V_0 + (q_0 + q_2)V_1 + (q_1 + q_2)V_2]
\end{align}

For neutral molecule ($q_0 + q_1 + q_2 = 0$, so $q_2 = -q_0 - q_1$):
\begin{align}
&= \frac{1}{2}[(q_0 + q_1)V_0 + (q_0 - q_0 - q_1)V_1 + (q_1 - q_0 - q_1)V_2] \\
&= \frac{1}{2}[(q_0 + q_1)V_0 - q_1 V_1 - q_0 V_2] \\
&= \frac{1}{2}[(q_0 + q_1)(V_0 - V_2) + q_1(V_0 - V_1 - V_2)]
\end{align}

This is \textbf{NOT zero} in general, even for uniform fields!

\textbf{Conclusion:} Bond-by-bond summation with Convention 1 does NOT reproduce the exact result.

\subsection{Why Bond Dipole Summation Fails}

The total molecular dipole is:
\begin{equation}
\vec{\mu}_{\text{total}} = \sum_i q_i \vec{r}_i
\end{equation}

The sum of bond dipoles (Convention 1) is:
\begin{align}
\sum_{\text{bonds}} \vec{\mu}_{ij}^{(1)} &= \sum_{\text{bonds}} (q_j - q_i)\frac{\vec{r}_j - \vec{r}_i}{2} \\
&\neq \vec{\mu}_{\text{total}}
\end{align}

These are \textbf{different quantities}! The sum of bond dipoles does not equal the total molecular dipole in general.

\section{Non-Uniform Electric Field: Approximate Result}

\subsection{Setup}
Electric field from point charges (environment):
\begin{equation}
\vec{E}(\vec{r}) = k \sum_{\alpha} \frac{Q_\alpha (\vec{r} - \vec{R}_\alpha)}{|\vec{r} - \vec{R}_\alpha|^3}
\end{equation}

Potential:
\begin{equation}
V(\vec{r}) = k \sum_{\alpha} \frac{Q_\alpha}{|\vec{r} - \vec{R}_\alpha|}
\end{equation}

This is \textbf{NOT} linear in $\vec{r}$, so $V(\vec{r}) \neq -\vec{F} \cdot \vec{r}$.

\subsection{Multipole Expansion}

Expand $V(\vec{r})$ around a reference point $\vec{r}_0$ (e.g., molecular center):
\begin{align}
V(\vec{r}) &\approx V(\vec{r}_0) + (\vec{r} - \vec{r}_0) \cdot \nabla V|_{\vec{r}_0} + \frac{1}{2}(\vec{r} - \vec{r}_0)^T \cdot [\nabla\nabla V]|_{\vec{r}_0} \cdot (\vec{r} - \vec{r}_0) + \cdots \\
&= V(\vec{r}_0) - (\vec{r} - \vec{r}_0) \cdot \vec{E}(\vec{r}_0) - \frac{1}{2}(\vec{r} - \vec{r}_0)^T \cdot [\nabla \vec{E}]|_{\vec{r}_0} \cdot (\vec{r} - \vec{r}_0) + \cdots
\end{align}

\subsection{Exact Energy (to Second Order)}
\begin{align}
E_{\text{exact}} &= \sum_i q_i V(\vec{r}_i) \\
&\approx \sum_i q_i \bigg[ V(\vec{r}_0) - (\vec{r}_i - \vec{r}_0) \cdot \vec{E}(\vec{r}_0) \nonumber \\
&\quad - \frac{1}{2}(\vec{r}_i - \vec{r}_0)^T \cdot [\nabla \vec{E}] \cdot (\vec{r}_i - \vec{r}_0) \bigg] \\
&= V(\vec{r}_0) \sum_i q_i - \vec{E}(\vec{r}_0) \cdot \sum_i q_i(\vec{r}_i - \vec{r}_0) \nonumber \\
&\quad - \frac{1}{2}\sum_i q_i (\vec{r}_i - \vec{r}_0)^T \cdot [\nabla \vec{E}] \cdot (\vec{r}_i - \vec{r}_0)
\end{align}

For a \textbf{neutral molecule} ($\sum_i q_i = 0$):
\begin{equation}
\boxed{E_{\text{exact}} \approx -\vec{E}(\vec{r}_0) \cdot \vec{\mu} - \underbrace{\frac{1}{2}\sum_i q_i (\vec{r}_i - \vec{r}_0)^T \cdot [\nabla \vec{E}] \cdot (\vec{r}_i - \vec{r}_0)}_{\text{quadrupole-field gradient term}}}
\end{equation}

where $\vec{\mu} = \sum_i q_i(\vec{r}_i - \vec{r}_0)$ is the dipole moment relative to $\vec{r}_0$.

\subsection{Dipole Approximation}
\begin{equation}
E_{\text{dipole}} \approx -\vec{E}(\vec{r}_0) \cdot \vec{\mu}
\end{equation}

\textbf{Error:}
\begin{equation}
\boxed{\Delta E = E_{\text{exact}} - E_{\text{dipole}} \approx -\frac{1}{2}\sum_i q_i (\vec{r}_i - \vec{r}_0)^T \cdot [\nabla \vec{E}] \cdot (\vec{r}_i - \vec{r}_0)}
\end{equation}

This is the \textbf{quadrupole-field gradient interaction}, which is \textbf{non-zero} for non-uniform fields!

\subsection{When is the Dipole Approximation Valid?}

The dipole approximation is good when:
\begin{equation}
\left| \frac{\text{quadrupole term}}{\text{dipole term}} \right| = \left| \frac{\sum_i q_i (\vec{r}_i - \vec{r}_0)^T \cdot [\nabla \vec{E}] \cdot (\vec{r}_i - \vec{r}_0)}{2\vec{E} \cdot \vec{\mu}} \right| \ll 1
\end{equation}

This occurs when:
\begin{itemize}
    \item Field gradient $|\nabla \vec{E}|$ is small
    \item Molecule size $|\vec{r}_i - \vec{r}_0|$ is small
    \item Quadrupole moment $\sum_i q_i (\vec{r}_i - \vec{r}_0)^T \otimes (\vec{r}_i - \vec{r}_0)$ is small
\end{itemize}

\section{E-field Definitions for Bond Formulation}

\subsection{E-field at Bond Midpoint}

For bond $(i,j)$ with midpoint $\vec{r}_{\text{mid}} = (\vec{r}_i + \vec{r}_j)/2$:

\begin{equation}
\vec{E}_{\text{mid}} = \vec{E}(\vec{r}_{\text{mid}})
\end{equation}

This is the \textbf{exact} E-field at the geometric center of the bond.

\subsection{E-field Averaged at Endpoints}

\begin{equation}
\vec{E}_{\text{avg}} = \frac{\vec{E}(\vec{r}_i) + \vec{E}(\vec{r}_j)}{2}
\end{equation}

Taylor expansion around midpoint:
\begin{align}
\vec{E}(\vec{r}_i) &\approx \vec{E}(\vec{r}_{\text{mid}}) - \frac{\vec{r}_j - \vec{r}_i}{2} \cdot [\nabla \vec{E}]|_{\text{mid}} \\
\vec{E}(\vec{r}_j) &\approx \vec{E}(\vec{r}_{\text{mid}}) + \frac{\vec{r}_j - \vec{r}_i}{2} \cdot [\nabla \vec{E}]|_{\text{mid}}
\end{align}

Therefore:
\begin{equation}
\vec{E}_{\text{avg}} = \vec{E}(\vec{r}_{\text{mid}}) + \mathcal{O}\left(|\vec{r}_j - \vec{r}_i|^2 \nabla^2 \vec{E}\right)
\end{equation}

For slowly varying fields, $\vec{E}_{\text{avg}} \approx \vec{E}_{\text{mid}}$ (first-order terms cancel).

\subsection{E-field Projection Along Bond}

Define bond direction: $\hat{n}_{ij} = \frac{\vec{r}_j - \vec{r}_i}{|\vec{r}_j - \vec{r}_i|}$

\textbf{Projection using averaged field:}
\begin{equation}
E_{\text{proj}} = \vec{E}_{\text{avg}} \cdot \hat{n}_{ij} = \frac{1}{2}[\vec{E}(\vec{r}_i) + \vec{E}(\vec{r}_j)] \cdot \hat{n}_{ij}
\end{equation}

\textbf{Projection using midpoint field:}
\begin{equation}
E_{\text{proj,mid}} = \vec{E}(\vec{r}_{\text{mid}}) \cdot \hat{n}_{ij}
\end{equation}

For uniform field: Both give $E_{\text{proj}} = \vec{F} \cdot \hat{n}_{ij}$ (exact).

For non-uniform field: Midpoint projection is more accurate.

\subsection{Relationship to Potential Difference}

The exact relationship between E-field and potential is:
\begin{equation}
V_j - V_i = -\int_{\vec{r}_i}^{\vec{r}_j} \vec{E}(\vec{r}) \cdot d\vec{r}
\end{equation}

For a straight-line path along the bond:
\begin{equation}
V_j - V_i \approx -\vec{E}(\vec{r}_{\text{mid}}) \cdot (\vec{r}_j - \vec{r}_i) + \mathcal{O}(|\vec{r}_j - \vec{r}_i|^2 \nabla \vec{E})
\end{equation}

Therefore:
\begin{equation}
\vec{E}(\vec{r}_{\text{mid}}) \cdot \hat{n}_{ij} \approx -\frac{V_j - V_i}{|\vec{r}_j - \vec{r}_i|}
\end{equation}

\section{Energy Differences Between States}

For the same molecule in two different environments (A and B):

\subsection{Exact Energy Difference}
\begin{equation}
\Delta E_{\text{exact}} = \sum_i q_i [V_B(\vec{r}_i) - V_A(\vec{r}_i)] = \sum_i q_i \Delta V_i
\end{equation}

\subsection{Dipole Approximation for Energy Difference}

If the field change is approximately uniform across the molecule:
\begin{equation}
\Delta E_{\text{dipole}} \approx -[\vec{E}_B(\vec{r}_0) - \vec{E}_A(\vec{r}_0)] \cdot \vec{\mu} = -\Delta \vec{E} \cdot \vec{\mu}
\end{equation}

\subsection{Bond Dipole Formulation for Energy Difference}

\begin{equation}
\Delta E_{\text{bond}} = \sum_{\text{bonds}} -\vec{\mu}_{ij} \cdot [\vec{E}_B(\vec{r}_{ij}) - \vec{E}_A(\vec{r}_{ij})]
\end{equation}

Even for energy differences, the bond formulation does NOT exactly equal the full calculation unless the field is uniform.

However, for \textbf{comparing relative trends}, systematic errors may partially cancel.

\section{Summary of Results}

\begin{table}[h]
\centering
\begin{tabular}{|p{3cm}|p{5cm}|p{5cm}|}
\hline
\textbf{Field Type} & \textbf{Dipole Formula} & \textbf{Accuracy} \\
\hline
Uniform & $E = -\vec{F} \cdot \vec{\mu}_{\text{total}}$ & \textbf{Exact} for total molecular dipole \\
\hline
 & $E = \sum_{\text{bonds}} -\vec{\mu}_{ij} \cdot \vec{F}$ & \textbf{Approximate} (bond dipoles $\neq$ total dipole) \\
\hline
Non-uniform (weak gradient) & $E \approx -\vec{E}(\vec{r}_0) \cdot \vec{\mu}_{\text{total}}$ & Good approximation (error $\sim$ quadrupole terms) \\
\hline
Non-uniform (strong gradient) & $E = \sum_i q_i V(\vec{r}_i)$ & \textbf{Exact} (must use full calculation) \\
\hline
\end{tabular}
\end{table}

\section{Recommendations}

\subsection{Ground Truth (Always Exact)}
\begin{equation}
\boxed{E_{\text{exact}} = \sum_i q_i V(\vec{r}_i)}
\end{equation}

Use this when:
\begin{itemize}
    \item Electric field is highly non-uniform (point charges nearby)
    \item High accuracy is required
    \item Computing absolute energies
\end{itemize}

\subsection{Dipole Approximation (Good for Uniform Fields)}
\begin{equation}
E_{\text{dipole}} = -\vec{E}(\vec{r}_0) \cdot \vec{\mu}_{\text{total}}
\end{equation}

Use this when:
\begin{itemize}
    \item Field is approximately uniform across the molecule
    \item Computing energy differences where quadrupole contributions cancel
    \item Interested in qualitative trends
\end{itemize}

\subsection{Bond Decomposition (Interpretive Tool)}
\begin{equation}
E_{\text{bonds}} = \sum_{\text{bonds}} -\vec{\mu}_{ij} \cdot \vec{E}(\vec{r}_{ij})
\end{equation}

Use this for:
\begin{itemize}
    \item Understanding which bonds contribute most to stabilization
    \item Qualitative chemical insights
    \item Relative comparisons (not absolute energies)
\end{itemize}

Note: This does NOT equal $\sum_i q_i V(\vec{r}_i)$ in general.

\section{Conclusion}

The dipole-field formulation $E = -\vec{\mu} \cdot \vec{E}$ is:
\begin{itemize}
    \item \textbf{Exact} for uniform fields using the total molecular dipole
    \item \textbf{Approximate} for non-uniform fields (misses quadrupole terms)
    \item \textbf{Not equivalent} to bond-by-bond summation in general
\end{itemize}

The exact formula $E = \sum_i q_i V(\vec{r}_i)$ is the ground truth and should be used for accurate calculations in non-uniform fields.

\end{document}

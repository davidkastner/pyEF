\documentclass[12pt]{article}
\usepackage{amsmath}
\usepackage{amssymb}
\usepackage{physics}
\usepackage{geometry}
\usepackage{xcolor}
\geometry{margin=1in}

\title{Line Integral vs Dot Product: Why Can't We Just Use $\vec{E}(\vec{r}) \cdot \vec{r}$?}
\author{}
\date{}

\begin{document}

\maketitle

\section{The Central Question}

Why do we need:
\begin{equation}
V(\vec{r}) = V_0 - \int_{\vec{r}_0}^{\vec{r}} \vec{E}(\vec{r}') \cdot d\vec{r}'
\end{equation}

instead of simply:
\begin{equation}
V(\vec{r}) \stackrel{?}{=} V_0 - \vec{E}(\vec{r}) \cdot \vec{r}
\end{equation}

\textbf{User's hypothesis}: If we allow $\vec{E}$ to be a function of the integration variable, could the formalism collapse to the same thing?

\section{The Fundamental Difference}

\subsection{Line Integral}
\begin{equation}
\int_{\vec{r}_0}^{\vec{r}} \vec{E}(\vec{r}') \cdot d\vec{r}'
\end{equation}

This integrates the field \textbf{along a path} from $\vec{r}_0$ to $\vec{r}$:
\begin{itemize}
    \item $\vec{r}'$ traces out the path
    \item $d\vec{r}'$ is an infinitesimal displacement along the path
    \item $\vec{E}(\vec{r}')$ is evaluated at each point on the path
    \item The integral sums up contributions from every point between $\vec{r}_0$ and $\vec{r}$
\end{itemize}

\subsection{Dot Product}
\begin{equation}
\vec{E}(\vec{r}) \cdot \vec{r}
\end{equation}

This evaluates the field \textbf{only at the endpoint} $\vec{r}$:
\begin{itemize}
    \item No path information
    \item Only knows the field at the final point
    \item Cannot account for field variations along the path
\end{itemize}

\section{Specific Example: Radial Field}

Let's examine a point charge field where we can compute everything exactly.

\subsection{Setup}
\begin{itemize}
    \item Point charge $Q$ at origin
    \item Electric field: $\vec{E}(\vec{r}) = k_e Q \frac{\vec{r}}{|\vec{r}|^3}$
    \item Potential: $V(\vec{r}) = k_e \frac{Q}{|\vec{r}|}$ (with $V(\infty) = 0$)
\end{itemize}

\subsection{Line Integral Approach}

Choose a straight-line path from $\infty$ to $\vec{r}$ along the radial direction.

Parameterize: $\vec{r}'(t) = t\hat{r}$ where $t$ goes from $\infty$ to $r = |\vec{r}|$

\begin{align}
V(\vec{r}) &= 0 - \int_{\infty}^{r} \vec{E}(t\hat{r}) \cdot d(t\hat{r}) \\
&= -\int_{\infty}^{r} k_e Q \frac{t\hat{r}}{t^3} \cdot \hat{r} \, dt \\
&= -\int_{\infty}^{r} k_e Q \frac{1}{t^2} dt \\
&= -k_e Q \left[-\frac{1}{t}\right]_{\infty}^{r} \\
&= k_e Q \frac{1}{r}
\end{align}

\textcolor{green}{\textbf{Correct!}} This gives us the right potential.

\subsection{Dot Product Approach}

Let's try $V(\vec{r}) \stackrel{?}{=} -\vec{E}(\vec{r}) \cdot \vec{r}$:

\begin{align}
-\vec{E}(\vec{r}) \cdot \vec{r} &= -k_e Q \frac{\vec{r}}{|\vec{r}|^3} \cdot \vec{r} \\
&= -k_e Q \frac{|\vec{r}|^2}{|\vec{r}|^3} \\
&= -k_e Q \frac{1}{|\vec{r}|} \\
&= -k_e \frac{Q}{r}
\end{align}

\textcolor{red}{\textbf{Wrong sign!}} And this is just one manifestation of the problem.

\subsection{Why Does This Fail?}

The dot product $\vec{E}(\vec{r}) \cdot \vec{r}$ gives:
\begin{equation}
\vec{E}(\vec{r}) \cdot \vec{r} = k_e Q \frac{\vec{r}}{|\vec{r}|^3} \cdot \vec{r} = k_e Q \frac{1}{|\vec{r}|}
\end{equation}

But the line integral gives:
\begin{equation}
\int_{\infty}^{\vec{r}} \vec{E}(\vec{r}') \cdot d\vec{r}' = -k_e Q \frac{1}{|\vec{r}|}
\end{equation}

The difference is not just a sign! The fundamental issue is that:
\begin{equation}
\boxed{\int_{\vec{r}_0}^{\vec{r}} \vec{E}(\vec{r}') \cdot d\vec{r}' \neq \vec{E}(\vec{r}) \cdot (\vec{r} - \vec{r}_0)}
\end{equation}

\section{Mathematical Reason: Fundamental Theorem}

\subsection{The Calculus Connection}

For a scalar function $f(x)$:
\begin{equation}
\int_a^b f'(x) \, dx = f(b) - f(a)
\end{equation}

This works because $f'(x)$ is the derivative of a single function.

\subsection{Why It Fails for $\vec{E} \cdot \vec{r}$}

The electric field satisfies:
\begin{equation}
\vec{E}(\vec{r}) = -\nabla V(\vec{r})
\end{equation}

This means:
\begin{equation}
\int_{\vec{r}_0}^{\vec{r}} \vec{E}(\vec{r}') \cdot d\vec{r}' = -\int_{\vec{r}_0}^{\vec{r}} \nabla V(\vec{r}') \cdot d\vec{r}' = -(V(\vec{r}) - V(\vec{r}_0))
\end{equation}

This is the gradient theorem (multivariable fundamental theorem of calculus).

But $\vec{E}(\vec{r}) \cdot \vec{r}$ is NOT the gradient of anything simple! Let's check:

\begin{align}
\vec{E}(\vec{r}) \cdot \vec{r} &\stackrel{?}{=} -\nabla V(\vec{r}) \cdot \vec{r} \\
&= -\frac{\partial V}{\partial x} x - \frac{\partial V}{\partial y} y - \frac{\partial V}{\partial z} z
\end{align}

This is NOT generally equal to $V(\vec{r})$ or any simple function of $V$.

\section{When DO They Match?}

\subsection{Uniform Field Case}

For $\vec{E}(\vec{r}) = \vec{F}$ (constant):

\textbf{Line integral:}
\begin{align}
\int_0^{\vec{r}} \vec{F} \cdot d\vec{r}' &= \vec{F} \cdot \int_0^{\vec{r}} d\vec{r}' \\
&= \vec{F} \cdot \vec{r}
\end{align}

\textbf{Dot product:}
\begin{equation}
\vec{F} \cdot \vec{r}
\end{equation}

\textcolor{green}{\textbf{They match!}}

This is the ONLY case where:
\begin{equation}
\int_0^{\vec{r}} \vec{E}(\vec{r}') \cdot d\vec{r}' = \vec{E}(\vec{r}) \cdot \vec{r}
\end{equation}

\subsection{Mathematical Condition}

The condition for equality is:
\begin{equation}
\int_0^{\vec{r}} \vec{E}(\vec{r}') \cdot d\vec{r}' = \vec{E}(\vec{r}) \cdot \vec{r}
\end{equation}

Taking the gradient of both sides:
\begin{equation}
\nabla \left[\int_0^{\vec{r}} \vec{E}(\vec{r}') \cdot d\vec{r}'\right] = \nabla[\vec{E}(\vec{r}) \cdot \vec{r}]
\end{equation}

Left side (by fundamental theorem):
\begin{equation}
\nabla \left[\int_0^{\vec{r}} \vec{E}(\vec{r}') \cdot d\vec{r}'\right] = \vec{E}(\vec{r})
\end{equation}

Right side:
\begin{align}
\nabla[\vec{E}(\vec{r}) \cdot \vec{r}] &= \vec{E}(\vec{r}) + (\nabla \vec{E}) \cdot \vec{r}
\end{align}

where $\nabla \vec{E}$ is the Jacobian matrix:
\begin{equation}
(\nabla \vec{E})_{ij} = \frac{\partial E_i}{\partial x_j}
\end{equation}

So we need:
\begin{equation}
\vec{E}(\vec{r}) = \vec{E}(\vec{r}) + (\nabla \vec{E}) \cdot \vec{r}
\end{equation}

This requires:
\begin{equation}
\boxed{(\nabla \vec{E}) \cdot \vec{r} = 0}
\end{equation}

This is satisfied when $\nabla \vec{E} = 0$, i.e., \textbf{uniform field}.

\section{Answering the User's Question}

\subsection{Can We Make Them Equivalent by Letting $\vec{E}$ Vary?}

\textbf{Question}: If we allow the E-field to also be a function of your integrating variable, could the formalism collapse to the same thing?

\textbf{Answer}: \textcolor{red}{\textbf{No}}, and here's why:

The line integral ALREADY has $\vec{E}$ as a function of the integration variable:
\begin{equation}
\int_{\vec{r}_0}^{\vec{r}} \vec{E}(\vec{r}') \cdot d\vec{r}'
\end{equation}

Here $\vec{r}'$ is the integration variable, and $\vec{E}(\vec{r}')$ varies along the path.

The problem is NOT that we need to make $\vec{E}$ vary. The problem is the \textbf{structure} of the integral itself:

\begin{center}
\begin{tabular}{|c|c|}
\hline
Line Integral & Dot Product \\
\hline
$\int \vec{E}(\vec{r}') \cdot d\vec{r}'$ & $\vec{E}(\vec{r}) \cdot \vec{r}$ \\
Integrates along path & Single evaluation \\
Path-dependent contributions & No path information \\
\hline
\end{tabular}
\end{center}

\subsection{Could We Rewrite the Energy?}

Let's try to express the energy in a different way.

\textbf{Standard:}
\begin{equation}
E = \int \rho(\vec{r}) V(\vec{r}) \, d^3r
\end{equation}

\textbf{Substitute line integral for $V$:}
\begin{align}
E &= \int \rho(\vec{r}) \left[V_0 - \int_{\vec{r}_0}^{\vec{r}} \vec{E}(\vec{r}') \cdot d\vec{r}'\right] d^3r \\
&= V_0 \int \rho(\vec{r}) \, d^3r - \int \rho(\vec{r}) \left[\int_{\vec{r}_0}^{\vec{r}} \vec{E}(\vec{r}') \cdot d\vec{r}'\right] d^3r
\end{align}

For neutral systems ($\int \rho \, d^3r = 0$):
\begin{equation}
E = -\int \rho(\vec{r}) \left[\int_{\vec{r}_0}^{\vec{r}} \vec{E}(\vec{r}') \cdot d\vec{r}'\right] d^3r
\end{equation}

This is a \textbf{double integral}: outer integral over charge distribution, inner integral over path.

\subsection{Why Can't We Simplify?}

We cannot simplify:
\begin{equation}
\int \rho(\vec{r}) \left[\int_{\vec{r}_0}^{\vec{r}} \vec{E}(\vec{r}') \cdot d\vec{r}'\right] d^3r
\end{equation}

to:
\begin{equation}
\int \rho(\vec{r}) [\vec{E}(\vec{r}) \cdot \vec{r}] \, d^3r
\end{equation}

because the inner integral depends on the ENTIRE path from $\vec{r}_0$ to $\vec{r}$, not just the endpoint field and position.

\section{Physical Interpretation}

\subsection{What Does $\vec{E} \cdot \vec{r}$ Represent?}

The quantity $\vec{E}(\vec{r}) \cdot \vec{r}$ can be interpreted as:
\begin{itemize}
    \item Component of the field in the direction of $\vec{r}$, scaled by $|\vec{r}|$
    \item For radial fields: $E_r \cdot r$
    \item NOT directly related to potential or energy
\end{itemize}

\subsection{What Does the Line Integral Represent?}

The line integral $\int \vec{E} \cdot d\vec{r}'$ represents:
\begin{itemize}
    \item Work done moving a unit charge along the path
    \item Accumulation of field contributions along the entire path
    \item Change in potential energy
    \item The potential difference (up to sign)
\end{itemize}

\subsection{Example: Why Path Matters}

Consider a point charge field and two charge distributions:

\textbf{Distribution 1}: Charge at $\vec{r} = (1, 0, 0)$
\begin{equation}
\vec{E}(\vec{r}) \cdot \vec{r} = k_e Q \frac{(1, 0, 0)}{1} \cdot (1, 0, 0) = k_e Q
\end{equation}

\textbf{Distribution 2}: Charge at $\vec{r} = (0, 1, 0)$
\begin{equation}
\vec{E}(\vec{r}) \cdot \vec{r} = k_e Q \frac{(0, 1, 0)}{1} \cdot (0, 1, 0) = k_e Q
\end{equation}

The dot products are the SAME! But the potentials are different:
\begin{align}
V((1,0,0)) &= k_e Q \\
V((0,1,0)) &= k_e Q
\end{align}

Wait, these are actually the same because both are at distance 1 from origin.

Let me try a better example:

\textbf{Distribution 1}: Charge at $\vec{r} = (2, 0, 0)$
\begin{equation}
\vec{E}(\vec{r}) \cdot \vec{r} = k_e Q \frac{(2, 0, 0)}{8} \cdot (2, 0, 0) = k_e Q \frac{4}{8} = \frac{k_e Q}{2}
\end{equation}

Potential:
\begin{equation}
V((2,0,0)) = k_e \frac{Q}{2}
\end{equation}

\textbf{Distribution 2}: Charge at $\vec{r} = (1, 0, 0)$
\begin{equation}
\vec{E}(\vec{r}) \cdot \vec{r} = k_e Q \frac{(1, 0, 0)}{1} \cdot (1, 0, 0) = k_e Q
\end{equation}

Potential:
\begin{equation}
V((1,0,0)) = k_e Q
\end{equation}

So:
\begin{itemize}
    \item At $(2,0,0)$: $\vec{E} \cdot \vec{r} = k_e Q/2$, $V = k_e Q/2$ (match!)
    \item At $(1,0,0)$: $\vec{E} \cdot \vec{r} = k_e Q$, $V = k_e Q$ (match!)
\end{itemize}

Hmm, for this radial field they're proportional! Let me check general radial:

\begin{align}
\vec{E} \cdot \vec{r} &= k_e Q \frac{\vec{r}}{|\vec{r}|^3} \cdot \vec{r} = k_e Q \frac{|\vec{r}|^2}{|\vec{r}|^3} = k_e Q \frac{1}{|\vec{r}|} \\
V(\vec{r}) &= k_e \frac{Q}{|\vec{r}|}
\end{align}

So for the Coulomb field: $\vec{E} \cdot \vec{r} = V$!

But this is a SPECIAL case! Let me try a different field.

\subsection{Better Example: Non-Radial Field}

Consider a dipole field, which is NOT purely radial.

Electric dipole at origin with moment $\vec{p} = p\hat{z}$:
\begin{equation}
\vec{E}(\vec{r}) = k_e \frac{1}{r^3} [3(\vec{p} \cdot \hat{r})\hat{r} - \vec{p}]
\end{equation}

Potential:
\begin{equation}
V(\vec{r}) = k_e \frac{\vec{p} \cdot \vec{r}}{r^3}
\end{equation}

Let's evaluate at a point, say $\vec{r} = (0, 0, z)$:

\textbf{Field:}
\begin{equation}
\vec{E}(0,0,z) = k_e \frac{1}{z^3} [3p\hat{z} - p\hat{z}] = k_e \frac{2p}{z^3} \hat{z}
\end{equation}

\textbf{Dot product:}
\begin{equation}
\vec{E} \cdot \vec{r} = k_e \frac{2p}{z^3} \hat{z} \cdot z\hat{z} = k_e \frac{2p}{z^2}
\end{equation}

\textbf{Potential:}
\begin{equation}
V(0,0,z) = k_e \frac{p \cdot z}{z^3} = k_e \frac{p}{z^2}
\end{equation}

\textcolor{red}{\textbf{Different!}} $\vec{E} \cdot \vec{r} = 2V$ for this case.

Now at the equator, $\vec{r} = (x, 0, 0)$:

\textbf{Field:}
\begin{equation}
\vec{E}(x,0,0) = k_e \frac{1}{x^3} [0 - p\hat{z}] = -k_e \frac{p}{x^3} \hat{z}
\end{equation}

\textbf{Dot product:}
\begin{equation}
\vec{E} \cdot \vec{r} = -k_e \frac{p}{x^3} \hat{z} \cdot x\hat{x} = 0
\end{equation}

\textbf{Potential:}
\begin{equation}
V(x,0,0) = k_e \frac{0}{x^3} = 0
\end{equation}

They match here, but only because both are zero!

\section{Summary}

\begin{enumerate}
    \item \textbf{Line integral is fundamental}: $V(\vec{r}) = V_0 - \int_{\vec{r}_0}^{\vec{r}} \vec{E}(\vec{r}') \cdot d\vec{r}'$

    \item \textbf{Dot product is NOT equivalent}: $\vec{E}(\vec{r}) \cdot \vec{r} \neq \int_0^{\vec{r}} \vec{E}(\vec{r}') \cdot d\vec{r}'$ in general

    \item \textbf{Why they differ}:
    \begin{itemize}
        \item Line integral accumulates contributions along the ENTIRE path
        \item Dot product only uses field at the endpoint
        \item Path information is essential for non-uniform fields
    \end{itemize}

    \item \textbf{When they match}: Only for uniform fields where $\vec{E} = \vec{F}$ (constant)
    \begin{equation}
    \int_0^{\vec{r}} \vec{F} \cdot d\vec{r}' = \vec{F} \cdot \vec{r}
    \end{equation}

    \item \textbf{Mathematical reason}: Requires $(\nabla \vec{E}) \cdot \vec{r} = 0$, satisfied only by uniform $\vec{E}$

    \item \textbf{Response to user's hypothesis}:
    \begin{itemize}
        \item The line integral ALREADY has $\vec{E}$ as a function of position
        \item Cannot collapse to a simple dot product for non-uniform fields
        \item The structure of the integral (path vs endpoint) is fundamentally different
    \end{itemize}

    \item \textbf{For energy calculations}:
    \begin{equation}
    \boxed{\int \rho V \, d^3r = -\int \rho(\vec{r}) \left[\int_0^{\vec{r}} \vec{E}(\vec{r}') \cdot d\vec{r}'\right] d^3r \neq -\int \rho(\vec{r}) [\vec{E}(\vec{r}) \cdot \vec{r}] \, d^3r}
    \end{equation}

    except for uniform fields!
\end{enumerate}

\end{document}
